\documentclass{mprop}
\usepackage{graphicx}

% alternative font if you prefer
%\usepackage{times}

% for alternative page numbering use the following package
% and see documentation for commands
%\usepackage{fancyheadings}


% other potentially useful packages
%\uspackage{amssymb,amsmath}
%\usepackage{url}
%\usepackage{fancyvrb}
%\usepackage[final]{pdfpages}

\begin{document}

%%%%%%%%%%%%%%%%%%%%%%%%%%%%%%%%%%%%%%%%%%%%%%%%%%%%%%%%%%%%%%%%%%%
\title{Exploring Machine Learning on EEG Signals for Predicting Central Neuropathic Pain in Spinal Cord Injury Patients}
\author{Cristian-Liviu Chirion}
\date{13 December 2019}
\maketitle
%%%%%%%%%%%%%%%%%%%%%%%%%%%%%%%%%%%%%%%%%%%%%%%%%%%%%%%%%%%%%%%%%%%

\educationalconsent
\newpage

%%%%%%%%%%%%%%%%%%%%%%%%%%%%%%%%%%%%%%%%%%%%%%%%%%%%%%%%%%%%%%%%%%%
\tableofcontents
\newpage
%%%%%%%%%%%%%%%%%%%%%%%%%%%%%%%%%%%%%%%%%%%%%%%%%%%%%%%%%%%%%%%%%%%

%%%%%%%%%%%%%%%%%%%%%%%%%%%%%%%%%%%%%%%%%%%%%%%%%%%%%%%%%%%%%%%%%%%
\section{Introduction}\label{intro}

\subsection{Motivation}

Central Neuropathic Pain (CNP) is a pathological condition that frequently appears in patients who have previously suffered Spinal Cord Injury (SCI). It manifests itself as a recurring pain, which has been described as a feeling of burning, stabbing, or electric shock~\cite{hulsebosch_mechanisms_2009}, and is often intense enough to interfere with a patient's daily routine and sleep, often leading to mental health issues and even suicide~\cite{hulsebosch_mechanisms_2009,vuckovic_prediction_2018}.

The condition, which is estimated to affect between 40\% and 80\% of SCI patients~\cite{hulsebosch_mechanisms_2009,vuckovic_prediction_2018}, is permanent, and a cure for it has not yet been found. There are various treatments available for CNP patients which can reduce the pain to tolerable levels. Antidepressants, for example, have been shown to be beneficial in mitigating neuropathic pain~\cite{finnerup_review_2008}, but they often cause significant side effects such as drowsiness, fatigue, bladder issues, digestive issues, which can further interfere with a patient's life~\cite{finnerup_review_2008,khawam_side_2006}.

Being able to accurately predict whether a patient is likely to develop CNP well in advance of the pain actually appearing would offer the opportunity to administer preventive treatments and could motivate new pharmacological studies for developing more efficient pain prevention medication.

\subsection{Prediction of CNP}

Electroencephalography (EEG) is a technique for monitoring and analysing brain activity by placing multiple sensors across a patient's scalp and using them to measure the intensity of power over time in various parts of the brain. The signal power readings from each individual sensor are referred to as a 'channel', each channel representing a different location on the scalp~\cite{noauthor_multi-channel_nodate}. There have been various studies that concluded that there could be statistically significant differences between the EEG signals of SCI patients without CNP and SCI patients who have developed or are about to develop CNP~\cite{vuckovic_prediction_2018}. This provides an opportunity for researching machine learning techniques to classify SCI patients based on whether they develop pain or not, thus making it possible to predict the onset of CNP.

%%%%%%%%%%%%%%%%%%%%%%%%%%%%%%%%%%%%%%%%%%%%%%%%%%%%%%%%%%%%%%%%%%%
\section{Statement of Problem}

The goal of this project is to create a machine learning model that can accurately classify SCI patients into one of two categories based on their EEG readings: patients who develop CNP and patients who don't. The algorithm will be trained using an existing data set of patients from both categories, consisting of the patients' EEG recordings. This data has been collected from SCI patients for a previous study, which analyzed the EEG activity of SCI patients with CNP, SCI patients without CNP, as well as able-bodied patients~\cite{vuckovic_dynamic_2014}.

For the purpose of the EEG recordings in our working data set, patients were asked to sit while looking at a screen. A readiness cue in the shape of a cross would appear on the screen, followed by a second cue after one second. The second cue was an instruction for participants to imagine one of three possible movements - a left arrow would indicate left hand movement, a right arrow would indicate right hand movement, and a down arrow would indicate movement of the legs. Participants were asked to keep imagining these movements for 3 seconds, and these trials were repeated approximately 50 times for each patient~\cite{vuckovic_dynamic_2014}.

%%%%%%%%%%%%%%%%%%%%%%%%%%%%%%%%%%%%%%%%%%%%%%%%%%%%%%%%%%%%%%%%%%%
\section{Background Survey}

It is important to take into account the various previous studies that have explored and analyzed EEG signals as a means of detecting neuropathic pain.

A good starting point in this sense is the analysis provided by \citet{vuckovic_dynamic_2014}, which has the clear advantage of having been performed on the same data set as the one we are working with. In this paper, the main criteria used as a basis of comparison between the different patient categories is the band power over multiple frequency bands, calculated based on the EEG readings. The study finds a statistically significant difference between the readings of patients with pain and patients without pain, noting that these differences are mostly localised within particular subsets of EEG channels, representing different locations across the patients' scalps. Consequently, we can expect that calculating the power over various frequency bands from our EEG data can produce useful features to use as input in a classifier, and that it will be useful to explore classification of the different individual channels in order to find the optimal subset of channels for classification.

\subsection{Data pre-processing}
\label{data-prep}

As with any other machine learning problem, one key aspect of our research is finding an efficient way to make the most of the available data by obtaining useful features through pre-processing. This is particularly important when it comes to classifying data sets of EEG recordings such as ours, since this type of data is often noisy and likely to contain outliers, but also because such data sets are often relatively small~\cite{lotte_review_2007}. Furthermore, EEG data consists of many readings of signal power over time, which are done over multiple channels, meaning that the resulting data set will be high-dimensional. \citet{lotte_review_2007} point out that in such situations we need to take into account the phenomenon known as the 'curse of dimensionality', meaning that the size of a data set required to train an accurate classifier increases exponentially with the number of dimensions in the training data. This means that in our problem it will likely be necessary to select a small subset of the available channels and/or to apply a pre-processing method that significantly reduces the number of dimensions while preserving any potentially useful features of the data.

The issue of noisy data also needs to be addressed. EEG recordings are highly susceptible to noise from both biological factors, such as blinks and muscle movements, as well as external factors such as radio or electrical interference~\cite{fitzgibbon_removal_2007}. It is therefore essential to apply noise removal methods so that a classification algorithm can be trained on the useful features of the data. For this purpose, \citet{shaker_eeg_2007} proposed a pipeline of multiple operations that can be applied on EEG readings for pre-processing, including filtering out frequencies that are outside the usual range of EEG signals (0 to 30 Hz), and applying a Discrete Wavelet Transform, which has been shown to be a useful technique for noise reduction~\cite{lang_noise_1996,lang_nonlinear_1995}.

For the purpose of analyzing the EEG data, it is common to estimate the power spectral density (PSD) of the signal over different frequency bands by applying a Fast Fourier Transform (FFT) to it~\cite{al-fahoum_methods_2014}. \citet{subasi_neural_2005} point out that there are four frequency bands which mostly contain the characteristic waveforms of EEG signals: delta (up to 4 Hz), theta (4 to 8 Hz), alpha (8 to 14  Hz), and beta (14 to 30  Hz). Analyzing the power of our EEG signals based on these four frequency bands would be consistent with the approach taken by \citet{vuckovic_dynamic_2014}, who use the same four bands as a basis for their calculations, although it should be noted that their definitions of the alpha and beta bands are slightly different (8 to 12 Hz and 16 to 24 Hz respectively). Applying this technique to our data set would have the advantage of drastically reducing its dimensionality, since the thousands of initial readings in each channel would be reduced to only one value per frequency band, while generating features that have been shown to be useful in previous studies~\cite{jarjees_causality_nodate,vuckovic_dynamic_2014,vuckovic_prediction_2018}.

Of course, there are other mathematical and statistical methods that are widely used for dimensionality reduction and thus could prove useful in our problem. One such technique is Principal Component Analysis (PCA), which reduces the number of dimensions while keeping a high variance along the axes, thus maintaining the differences between various data points~\cite{wold_principal_1987,kumar_understanding_2018}. Other methods include random projections, which involve projecting the data onto a smaller number of dimensions while maintaining data variance~\cite{bingham_random_2001}, and feature agglomeration, which uses clustering to find features that are similar to each other and merge them~\cite{wang_towards_2018}.

It should be noted that any approach to data processing entails a risk of information loss if not executed in an optimal manner, and that care must be taken to preserve useful features in the data when applying transformations to it. For example, \citet{klonowski_everything_2009} points out that methods involving Fourier Transforms are not always highly accurate when applied to EEG signals, and suggests non-linear approaches such as Higuchi's fractal dimension method as an alternative. Proposed by \citet{higuchi_approach_1988}, this method involves analyzing the EEG signal in the time domain, which represents the default form of EEG data, rather than a frequency domain such as the one generated by a Fourier Transform. Although we need to acknowledge and consider its limitations, such as its sensitivity to noise and its narrow interval of possible values, Higuchi's method has been shown to yield accurate results in previous EEG analysis studies, especially if the signal had previously been split into smaller segments~\cite{kesic_application_2016}. This is, therefore, a method that we should consider when processing data for this project.

A more simplistic, but potentially worthwhile, way of handling the data would be to use the raw EEG signal, after a noise reduction algorithm has been applied to it, and feed it into a classifier without any other processing. This means that the resulting machine learning model would be trained on many signal amplitude readings over time for a subset of channels. Although this approach would have the advantage of preserving all the information from the original data, the large amount of input for each data point would entail a high risk of overfitting the classifier, and choosing an optimal subset of channels would be an essential, and potentially difficult task. Nevertheless, this approach has been successfully used in studies such as the one conducted by \citet{kaper_bci_2004}, and is worth keeping into account as an option.

When it comes to data pre-processing, it is clear that no individual method yields prefect results, thus a process of trial and error might be the best way to determine which approach is best for a particular problem. Ultimately, a good approach for our study would be to try some or all of the methods described above and to evaluate their performance in order to identify the one with the best results.
 
\subsection{Training a classifier}

A key factor in the quality of results obtained in a classification problem such as ours is choosing a classification algorithm that suits the data set we are working with. We therefore analyze a number of previous studies that involved applying machine learning classifiers to EEG data in order to find out which classifiers have been seen to perform well in problems similar to ours.

\citet{lotte_review_2007} points out that Support Vector Machines (SVM) is a linear method that has been used successfully in a number of EEG classification problems, often outperforming other classifiers, in part thanks to its ability to classify high-dimensional data and relatively small data sets. The algorithm used by SVM involves representing the training data points in a multi-dimensional vector space and finding a hyperplane that separates the data points from different classes~\cite{gandhi_support_2018}. Examples of studies on EEG that obtained good results from SVM classification include \citet{vuckovic_prediction_2018}, who obtained a 76\% accuracy, and \citet{kaper_bci_2004}, who obtained a 84.5\% accuracy. However, \citet{gallardo_transferable_2017} note that SVM required a large subset of channels to perform well, which could make it more difficult to fit our data set into the classifier.

A method that is somewhat similar to SVM is Linear Discriminant Analysis (LDA), which is also a linear algorithm which aims to find a hyperplane that separates the two classes, the difference being that, unlike SVM, LDA achieves this by projecting the data on vectors that maximize the distance between the two classes~\cite{lotte_review_2007}. Since this approach is based on the variance between data points and it results in finding a subset of representative features, it is also similar to the PCA technique described in section \ref{data-prep}, although LDA is used to find the features that best discriminate between the classes, while PCA does not take class separation into account~\cite{martinez_pca_2001}. In the study conducted by \citet{vuckovic_prediction_2018}, a classifier based on LDA achieved an accuracy of 77\%, which was slightly better than the result of SVM. However, \citet{lotte_review_2007} points out that, due to its linearity, LDA is likely to perform poorly as the input data becomes more complex, an aspect that was observed in practice by \citet{gallardo_transferable_2017}, who note that accuracy under LDA decreased significantly as the number of channels increased.

Another method that could prove useful in our research is the Naive Bayes Classifier (NBC), which relies on Bayes' probability theorem to calculate a 'maximum a posteriori', i.e. the class that an element is most likely to belong to based on its feature vector~\cite{rish_empirical_2001}. This type of classification algorithm has the advantage of having a good performance expectation when there are only two possible classes of data, due to its low expected classification error~\cite{rish_empirical_2001}. It has been known to perform well in many different machine learning problems~\cite{rish_empirical_2001, chen_automated_2012}, including EEG signal classification~\cite{vuckovic_prediction_2018}, often resulting in a high accuracy, although we also need to consider that NBC is not typically used with high-dimensional input data~\cite{chen_automated_2012}, and that the theory behind this classifier assumes that the input features are independent from each other, which is not guaranteed in EEG data~\cite{gallardo_transferable_2017}.

neural nets

conclusion


%%%%%%%%%%%%%%%%%%%%%%%%%%%%%%%%%%%%%%%%%%%%%%%%%%%%%%%%%%%%%%%%%%%
\section{Proposed Approach}

state how you propose to solve the software development problem. Show that your proposed approach is feasible, but identify any risks.

%%%%%%%%%%%%%%%%%%%%%%%%%%%%%%%%%%%%%%%%%%%%%%%%%%%%%%%%%%%%%%%%%%%
\section{Work Plan}

show how you plan to organize your work, identifying intermediate deliverables and dates.

%%%%%%%%%%%%%%%%%%%%%%%%%%%%%%%%%%%%%%%%%%%%%%%%%%%%%%%%%%%%%%%%%%%
% it is fine to change the bibliography style if you want
\bibliographystyle{plainnat}
\bibliography{references}
\end{document}
